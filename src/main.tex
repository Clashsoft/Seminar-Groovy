%%
%% Author: adriankunz
%% 2018-12-28
%%

% =============== Preamble ===============

% --------------- Definitions ---------------
\newcommand{\paperTitle}[0]{Die Sprache Groovy}
\newcommand{\paperSubtitle}[0]{Seminar Skriptsprachen, Wintersemester 2018/19}
\newcommand{\paperDate}[0]{\today}
\newcommand{\paperKeywords}[0]{Groovy, Seminar, Skriptsprachen}
\newcommand{\paperAuthor}[0]{Adrian Kunz}

% --------------- Commands ---------------
\newcommand{\code}[1]{\mintinline{groovy}{#1}}

\newcommand{\codelisting}[3]{
\begin{listing}[ht]
	\inputminted{#1}{#1/#2}
	\caption{#3}
	\label{lst:#2}
\end{listing}
}

% --------------- Document Setup ---------------
\documentclass[11pt,a4paper]{article}
\usepackage[left=2.5cm,right=2.5cm,top=2.5cm,bottom=2cm]{geometry}

\title{\paperTitle}
\author{\paperAuthor}
\date{\paperDate}

% --------------- Packages ---------------
\usepackage[ngerman]{babel}
\usepackage[T1]{fontenc}
\usepackage[utf8]{inputenc}
\usepackage{url}

% Minted
\usepackage{minted}
\usemintedstyle{vs}
\setminted{frame=single,tabsize=2,linenos}

% Hyperref
\usepackage{hyperref}
\hypersetup{
pdftitle={\paperTitle{}},
pdfauthor={\paperAuthor{}},
pdfsubject={\paperTitle{}},
pdfkeywords={\paperKeywords},
bookmarksnumbered=true,
bookmarksopen=true,
hidelinks,
}
\usepackage{hypcap}

% =============== Document ===============

\begin{document}
	% --------------- ---------------

	\maketitle

	% --------------- ---------------

	\section{Einleitung}\label{sec:einleitung}

	% TODO

	% --------------- ---------------

	\section{Sprache}\label{sec:sprache}

	Im folgenden Abschnitt wird Groovy als Sprache vorgestellt.
	Die Syntax ist stark auf Java und andere C-ähnliche Sprachen angelehnt.
	Deshalb wird im Unterabschnitt~\ref{subsec:vergleichMitJava} durchgeführt ein Vergleich mit Java durchgeführt.
	Dabei wird sowohl auf syntaktischer als auch auf semantischer Ebene Bezug genommen.
	In Unterabschnitt~\ref{subsec:besonderheiten} werden neue Sprachelemente und Besonderheiten dargestellt.

	Bei Groovy handelt es sich {u.a.} um eine Skriptsprache.
	Somit sollten Groovy-Programme einfach auszuführen sein, möglichst ohne Zwischenschritte wie etwa ein Aufruf eines Compilers.
	Groovy-Skripte können entsprechend mit dem \code{groovy}-Programm ausgeführt werden.
	Wie etwa bei Bash oder Python können Groovy-Dateien (\code{.groovy}) aber auch mit einer Shebang-Zeile beginnen, um sie direkt ausführbar zu machen.
	Variablen und Statements können direkt in der obersten Ebene platziert werden, ebenso Imports, Klassen und Methoden.

	\codelisting{groovy}{hello-world.groovy}{Hello World in Groovy}

	Das Listing~\ref{lst:hello-world.groovy} kann mit dem Befehl \code{groovy hello-world.groovy} ausgeführt werden;
	es kann durch das Shebang aber auch direkt als \code{./hello-world.groovy} aufgerufen werden\footnote{Dafür muss der Programmcode in einer Datei namens \code{hello-world.groovy} gespeichert werden;
	der Aufruf via Shebang setzt voraus, dass diese ausführbar ist.}.
	Die Ausgabe ist in beiden Fällen \code{Hello World!}.
	Auffällig ist das Schlüsselwort \code{def}, der Methodenaufruf von \code{println} ohne Klammern und die Zeichenkette mit Interpolierung.
	Auf diese wird entsprechend in den Unterabschnitten~\ref{subsubsec:dynamische-typisierung},~\ref{subsubsec:optionale-syntax} und~\ref{subsubsec:strings} näher eingegangen.

	Zu beachten ist, dass es sich zwar um ein Skript handelt, aber Groovy dennoch in jedem Fall vor der Ausführung kompiliert wird.
	Bei Skripts geschieht dieser Schritt jedoch ohne dass zusätzliche Dateien gespeichert werden;
	der generierte Bytecode wird nur im Arbeitsspeicher zwischengespeichert und am Ende der Ausführung verworfen.

	\subsection{Vergleich mit Java}\label{subsec:vergleichMitJava}

	Da ein Großteil der Sprachelemente und deren Syntax direkt übernommen wurde, ist es möglich, viele Java-Programme direkt als Groovy-Programm zu verwenden.
	Dies hat einerseits den Vorteil, dass bei der Überführung eines Projekts von Java zu Groovy zunächst nur wenige Änderungen vorgenommen werden müssen.
	Andererseits macht es das Erlernen der Sprache für Entwickler mit einem Java-Hintergrund einfacher, da sie direkt gültigen Groovy-Code schreiben können und sich mit der Zeit spezifische Konzepte aneignen können.

	Es gibt jedoch einige Fälle, in denen ein Java-Programm nicht auch ein Groovy-Programm ist.
	Triviale Beispiele sind jene Programme, in denen Schlüsselwörter von Groovy als Bezeichner verwendet werden.
	Zu den Kontrollstrukturen, die Groovy in Version 2 nicht unterstützt, gehören Do-While-Schleifen, Try-With-Resource- Blöcke und Block-Statements.
	Die Java-Syntax für Lambda-Ausdrücke, Methoden-Referenzen und Array-Initialisierung werden von dieser Version auch nicht unterstützt.
	In Groovy 3~\cite{groovy-lang:release3}, welches sich derzeit in Entwicklung befindet, werden jedoch mithilfe eines neuen Parsers Bemühungen betrieben, auch diese Sprachelemente zu erlauben und somit den Umstieg weiter zu erleichtern.

	Unterschiede können auch bei Programmen auftreten, die in beiden Sprachen syntaktisch gültig sind, da sich in einigen Fällen die Semantik in Groovy anders als in Java verhält.
	Dazu gehören das implizite Importieren von Pakete zusätzlich zu \code{java.lang}, dazu gehören {u.a.} \code{java.io}, \code{java.net}, \code{java.util}, \code{groovy.lang} und \code{groovy.util}, sowie die Klassen \code{java.math.BigInteger} und \code{java.math.BigDecimal}.
	Weiterhin verhalten sich innere und geschachtelte Klassen, String- und Char-Literale und Typkonvertierungen zwischen primitiven Typen, Wrapper-Typen, \code{BigInteger} und \code{BigDecimal} anders als in Java.

	Weitere Informationen und Beispiele zu den oben genannten Unterschieden sind bei~\cite{groovy-lang:differences} zu finden.

	\subsubsection{Dynamische Typisierung}\label{subsubsec:dynamische-typisierung}

	% TODO

	\subsubsection{Prototyp-Erweiterung}

	% TODO

	\subsubsection{Multi-Methoden}

	Unter Multi-Methoden versteht man eine besondere Art der Semantik von Methodenaufrufen.
	Es handelt sich um eine Form des Polymorphismus, bei der die aufzurufende Methode nicht nur anhand des Laufzeittyps des Empfängers, sondern auch anhand der Typen der Argumente ausgewählt wird.
	In Java werden Methodenaufrufe, bei denen der Aufruf vom Laufzeittyp des Empfängers abhängt, als \emph{virtuell} bezeichnet.
	Dabei ist das \emph{Überschreiben} von Methoden in Unterklassen von Bedeutung.
	Orthogonal dazu gibt es die Möglichkeit, Methoden zu \emph{überladen}.
	Dies bezeichnet das Deklarieren von zwei oder mehr Methoden im gleichen Kontext, die denselben Namen besitzen und sich in der Anzahl der Parameter oder deren Typen unterscheiden.
	Beim Aufruf dieser Methoden wird abhängig von den zur Kompilierungszeit bestimmten Typen der Argument ausgewählt, welche Methode aufgerufen wird.
	Zur Laufzeit sind die Typen der Argument für die Auswahl der Methode dann nicht mehr relevant.

	\codelisting{groovy}{multi-methods.groovy}{Multi-Methoden in Groovy}

	Zur Verdeutlichung des Verhaltens von Groovy soll Listing~\ref{lst:multi-methods.groovy} betrachtet werden.
	Wenn das Programm ausgeführt wird, ist die Assertion erfolgreich und es wird nichts ausgegeben.
	Grund dafür ist, dass die \code{take}-Methode aufgerufen wird, welche einen \code{String} erwartet.
	Der Laufzeittyp des Arguments \code{o} ist \code{String}, während der statische Typ \code{Object} ist.
	In Java würde die Methodenauswahl zur Kompilierungszeit erfolgen und die \code{take(Object)}-Variante bevorzugen, da dort beim Überladen nur der statische Typ relevant ist.

	\subsection{Besonderheiten}\label{subsec:besonderheiten}

	% TODO

	\subsubsection{Zahlen}\label{subsubsec:zahlen}

	Mathematische Konzepte haben eine besondere Bedeutung in Groovy.
	Die Klassen \code{BigInteger} und \code{BigDecimal} sind standardmäßig importiert~\cite[{Abs.} 1.]{groovy-lang:differences} und leichter zu verwenden als in Java, da sie übliche Operatoren unterstützen.
	Um Fließkommafehler zu vermeiden, sind Literale mit Nachkommastellen ohne explizites Suffix oder Cast vom Typ \code{BigDecimal}~\cite[{Abs.} 5.2.]{groovy-lang:syntax}.
	Das Ergebnis von Divisionen ist in vielen Fällen\footnote{wenn einer der beiden Operanden ein \code{float} oder \code{double} ist, ist das Ergebnis vom Typ \code{double}} auch ein \code{BigDecimal}~\cite[{Abs.} 5.5.1.]{groovy-lang:syntax}.
	Der Operator \code{**} kann verwendet werden, um die Potenz einer Zahl zu berechnen~\cite[{Abs.} 5.5.2.]{groovy-lang:syntax}.

	Übliche Operatoren können auch von eigenen Klassen verwendet werden, indem sie Methoden mit besonderen Namen wie \code{plus} für \code{+}, \code{leftShift} für \code{<<} oder \code{next} für \code{++} deklarieren~\cite[{Abs.} 10.]{groovy-lang:operators}.
	Dies macht es möglich, Klassen für mathematische Objekte wie komplexe Zahlen, Brüche oder Polynome zu erstellen und mit Operatoren zu verwenden.
	Eigene Operatoren können jedoch nicht definiert werden.

	In Groovy verhalten sich primitive Datentypen zu Kompilierungs- und Laufzeit wie Objekte ihrer Wrapper-Klassen~\cite[{Abs.} 10.]{groovy-lang:differences}.
	Somit unterstützen sie die üblichen Funktionalitäten, die sich auf Objekte beziehen.

	\subsubsection{Zeichenketten}\label{subsubsec:strings}

	Auch Zeichenketten sind vielfältiger als in Java.
	Es existieren sechs Arten von String-Literalen in Groovy:

	\begin{itemize}
		\item \code{'single quoted'}, unterstützt keine Interpolation oder Zeilenumbrüche,
		\item \code{'''triple single quoted'''}, unterstützt keine Interpolation aber mehrere Zeilen,
		\item \code{"double quoted"}, unterstützt Interpolation mit dem \code{$}-Zeichen, aber nur einzeilig,
		\item \code{"""triple quoted"""}, unterstützt Interpolation und mehrere Zeilen,
		\item \code{/slashy/}, für reguläre Ausdrücke, Interpolation und Zeilenumbrüche werden auch unterstützt, und
		\item \code{$/dollar slashy/$}, ähnlich zu \code{/slashy/} aber mit besonderen Escape-Regeln.
	\end{itemize}

	Besonders interessant ist hierbei die Interpolation, welche ein einfaches Einsetzen von Werten in Strings erlaubt.
	Das Beispiel~\ref{lst:string-interpolation.groovy} zeigt, wie diese in String-Literalen zum Einsatz kommt.
	\codelisting{groovy}{string-interpolation.groovy}{String-Interpolation in Groovy}
	Im ersten String-Literal in Zeile 3 wird anstelle von \code{$a} und \code{$b} wird der Wert der Variablen eingesetzt;
	\code{${a+b}} wertet den Ausdruck \code{a+b} aus und setzt das Ergebnis in den resultierenden String ein.

	Zu beachten ist hier, dass es sich bei dem linken String-Literal in Zeile 3 eigentlich nicht um einen \code{String} handelt, sondern um eine Instanz der Klasse \code{GString}.
	Diese speichert auch die Objekte, die anstelle der Platzhalter eingesetzt werden.
	Erst der Vergleich mit \code{==} verursacht eine Konvertierung zu \code{String}; dabei werden auch Ersetzung vorgenommen.

	In Zeile 5 wird ein GString verwendet, der eine \code{Closure}-Instanz als Platzhalter verwendet.
	Diese wird jedes Mal aufgerufen, wenn der GString zu einem String konvertiert wird.
	Dabei können auch Seiteneffekte auftreten und der Rückgabewert kann sich ändern, was im Beispiel zur Ausgabe \code{0 1 0 3} führt.
	Auf Closures wird in Unterabschnitt~\ref{subsubsec:closures} näher eingegangen.

	\subsubsection{Listen und Maps}

	Auch Listen und Maps haben in Groovy erste Bürgerrechte und einen großen Satz von Funktionen, um ihre Verwendung zu erleichtern.

	\codelisting{groovy}{lists-and-maps.groovy}{Listen und Maps in Groovy}

	Listing~\ref{lst:lists-and-maps.groovy} zeigt einige Möglichkeiten, Listen und Maps zu verwenden.
	In Zeile 2 wird zunächst eine Liste mit der \code{[]}-Syntax angelegt;
	diese ist ohne explizite Typangabe eine \code{ArrayList}.
	In Zeile 4 wird der Index-Operator verwendet, um auf das erste und letzte Element zuzugreifen, dabei ist die Verwendung negativer Zahlen wie etwa in Python möglich.
	Danach wird mit dem \code{<<}-Operator ein Element ans Ende der Liste angehängt.
	Die 6. Zeile zeigt, dass der Index-Operator auch mehrere Argument sowie Intervalle unterstützt.
	Der \code{..}-Operator sowie der nicht-inklusive \code{..<}-Operator sind auch in anderen Kontexten verwendbar, um einfach Intervalle zu definieren, beispielsweise in Foreach-Schleifen.

	In Zeile 8 wird eine Map angelegt, zu erkennen an den Schlüssel-Wert-Paaren in den eckigen Klammern.
	Bei den Bezeichnern \code{a} und \code{b} handelt es sich nicht um Variablennamen, stattdessen wird hier die Zeichenkette \code{a} bzw. \code{b} als Schlüssel verwendet.
	Das erzeugte Objekt verwendet die \code{LinkedHashMap}-Implementierung, damit die Reihenfolge der Schlüssel erhalten bleibt.
	Zeile 10 zeigt, wie der Zugriff auf den Wert eines Schlüssels sowohl mit dem Index-Operator als auch mit dem Punkt-Operator möglich ist, als wäre \code{d} eine Klassenvariable.
	In der nächsten Zeile werden auch beide Möglichkeiten verwendet, um neue Einträge hinzuzufügen.
	Auffällig ist der Zugriff auf unbekannte Schlüssel in Zeile 12, welcher auch mit dem Punkt-Operator möglich ist und \code{null} zurückgibt.

	Andere Collections und Arrays können durch Coercion von Listen- oder Map-Literalen erzeugt werden.
	Weitere Informationen zu Arrays, Listen und Maps sind in~\cite[{Abs.} 7.-9.]{groovy-lang:syntax} zu finden.

	\subsubsection{Null-Sicherheit}

	Obwohl Groovy im Kern eine Skriptsprache ist und Sicherheit daher eine kleinere Rolle spielt, ist es dennoch im Bereich Null-Sicherheit besser ausgestattet als Java.
	Im Kern stehen dabei zwei sehr nützliche Operatoren, \code{?.} und \code{?:}, welche im Folgenden kurz vorgestellt werden.

	Der Safe-Navigation-Operator \code{?.} wird verwendet, um \code{NullPointerException}s zu vermeiden.
	Anstelle des \code{.}-Operators für Methoden- und Feldzugriffe sorgt er dafür, dass das Ergebnis des Zugriffs \code{null} ist, wenn der Empfänger \code{null} ist.
	Somit ist beispielsweise der Wert von \code{person?.name} gleich \code{null}, wenn \code{person} schon \code{null} ist.

	Der \code{?:}-Operator, auch Elvis-Operator genannt, ist ein syntaktischer Zucker der Ausdrücke der Form \code{lhs ?: rhs} durch \code{lhs ? lhs : rhs} ersetzt.
	Dies erscheint im ersten Moment nicht nützlich, doch durch die Regeln der Groovy Truth~\cite[{Abs.} 5.]{groovy-lang:semantics}, welche die implizite Konvertierung zu Bool-Werten bestimmen, ist er sehr mächtig.
	Diese sorgen nämlich dafür, dass \code{rhs} vom Ternary-Operator ausgewählt wird, wenn \code{lhs} \code{null} ist, und \code{lhs} wenn nicht.
	Jedoch kann dies auch zu Problemen in der Semantik führen:
	Leere Strings, Listen und Maps oder Zahlen mit dem Wert \code{0} gelten neben \code{null}-Referenzen auch als \code{falsy}, weshalb das Ergebnis des Elvis-Operators mit ihnen auf der linken Seite auch die rechte Seite zurückgibt.
	In der Praxis ist dies jedoch weniger ein Problem, da Variablen mit eben diesen Datentypen, welche gleichzeitig auch \code{null} werden können, aufgrund der Existenz geeigneterer ``Leer''-Werte kein guter Stil sind.

	\subsubsection{Optionale Syntax}\label{subsubsec:optionale-syntax}

	% TODO

	\subsubsection{Closures}\label{subsubsec:closures}

	% TODO

	\subsubsection{AST-Transformation}

	% TODO

	% --------------- ---------------

	\section{Community, Einsatz und Verbreitung}\label{sec:community}

	% TODO

	\subsection{Gradle}\label{subsec:gradle}

	% TODO

	\subsection{Grails}\label{subsec:grails}

	% TODO

	\subsection{Griffon}\label{subsec:griffon}

	% TODO

	% --------------- ---------------

	\section{Zusammenfassung}\label{sec:zusammenfassung}

	% TODO

	% --------------- ---------------

	\bibliography{main}
	\bibliographystyle{ieeetr}

	% --------------- ---------------
\end{document}
