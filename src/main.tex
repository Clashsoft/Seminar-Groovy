%%
%% Author: adriankunz
%% 2018-12-28
%%

% =============== Preamble ===============

% --------------- Definitions ---------------
\newcommand{\paperTitle}[0]{Groovy}
\newcommand{\paperSubtitle}[0]{Seminar Skriptsprachen, Wintersemester 2018/19}
\newcommand{\paperDate}[0]{\today}
\newcommand{\paperKeywords}[0]{Groovy, Seminar, Skriptsprachen}
\newcommand{\paperAuthor}[0]{Adrian Kunz}

% --------------- Document Setup ---------------
\documentclass[a4paper]{article}
\usepackage[left=2.5cm,right=2.5cm,top=2.5cm,bottom=2cm]{geometry}

\title{\paperTitle}
\author{\paperAuthor}
\date{\paperDate}

% --------------- Packages ---------------
\usepackage[ngerman]{babel}
\usepackage[T1]{fontenc}
\usepackage[utf8]{inputenc}
\usepackage{url}

% Minted
\usepackage{minted}
\usemintedstyle{vs}
\setminted{frame=single,tabsize=2,linenos}

\newmintinline[code]{groovy}{}
\newmintinline[plain]{text}{breaklines}

\newcommand{\outquote}[1]{``{#1}''}

\newcommand{\codelisting}[3]{\begin{listing}[htp]
	\inputminted{#1}{#1/#2}
	\vspace{-3ex}
	\caption{#3}
	\label{lst:#2}
\end{listing}}

% Hyperref
\usepackage{hyperref}
\hypersetup{
pdftitle={\paperTitle{}},
pdfauthor={\paperAuthor{}},
pdfsubject={\paperTitle{}},
pdfkeywords={\paperKeywords},
bookmarksnumbered=true,
bookmarksopen=true,
hidelinks,
}
\usepackage{hypcap}

% =============== Document ===============

\begin{document}
% --------------- ---------------

\maketitle

% --------------- ---------------

\section{Einleitung}\label{sec:einleitung}

Jedem Entwickler, der schon einmal Java-Programme geschrieben hat, ist vermutlich aufgefallen, wie diese selbst für einfachste Aufgaben sehr umfangreich werden können.
Ein Hello-World-Programm muss in einer Methode in einer Klasse stehen;
Datenmodellklassen sind mit Gettern und Settern überfüllt;
Dateizugriffe müssen mit Try/Catch-Blöcken geschützt werden.
Für Skripte, welche per Definition sehr kurz sind, ist es daher wenig geeignet.

Dennoch bringt die Entwicklung mit dem Java-Ökosystem einige Vorteile:
Es bietet ein Rahmenwerk für portable und sichere Programme, bei deren Entwicklung man sich auf eine große Community und viele Bibliotheken verlassen kann.
Bemühungen, diese Vorteile mit Skriptsprachen zu verbinden, bildeten Projekte wie JRuby, Jython und Rhino, welche Ruby, Python und JavaScript in der Java Virtual Machine (JVM) lauffähig machten.
Keine dieser Sprachen war jedoch eigens auf Java ausgelegt oder einfach zu verwenden.

Deshalb wurde im Jahr 2003 die Skriptsprache \emph{Groovy} von Softwareentwickler und Java-Enthusiast James Strachan mit einem Blogpost~\cite{james-strachan-blog} angekündigt.
Sie sollte dynamisch typisiert sein, aber für die JVM kompiliert werden und somit bestehende Java-Bibliotheken und -Code nutzen können.
Primär basierend auf Java, sollten auch ausgewählte Konzepte von Python und Ruby übernommen werden.
Design-Ziel der neuen Sprache war und ist bis heute, Java-Entwicklern das Leben zu vereinfachen.
Als Anwendung für eine kompakte und ausdrucksstarke Syntax war Unit Testing geplant, was traditionell aufwändig war.

Im März 2004 wurde Groovy dem Java Community Process unter Java Specification Request 241 eingereicht~\cite{jcp:jsr241}.
Nach einigen Vorgängerversionen und der Standardisierung wurde schließlich Version 1.0 am 2.~Januar 2007 veröffentlicht.
Mit v2.0 wurde 2012 die Unterstützung für statische Typüberprüfung und Kompilierung eingeführt.
Das Groovy-Projekt wurde nach einer Inkubationsphase am 18.~November 2015 von der Apache Foundation aufgenommen~\cite{apache-incubator:groovy}.
Die aktuelle stabile Version ist 2.5.5, welche am 24.~Dezember 2018 erschien; parallel dazu wird aber schon an Version 3 der Sprache gearbeitet.

Diese Arbeit ist der Skriptsprache Groovy gewidmet.
Die Eignung für Domain-Specific Languages (DSLs) soll im Folgenden besondere Aufmerksamkeit erhalten.
Dazu werden zunächst in Abschnitt~\ref{sec:sprache} die Grundlagen der Sprache erläutert und deren Anwendung für DSLs gezeigt.
Danach wird das Ökosystem um Groovy sowie zwei der wichtigsten Frameworks, welche sich die Sprachbesonderheiten zunutze machen, in Abschnitt~\ref{sec:community} betrachtet.
Eine Zusammenfassung folgt in Abschnitt~\ref{sec:zusammenfassung}.

% --------------- ---------------

\section{Sprache}\label{sec:sprache}

Im folgenden Abschnitt wird Groovy als Sprache vorgestellt.
Die Syntax ist stark an Java und andere C-ähnliche Sprachen angelehnt.
Deshalb wird im Unterabschnitt~\ref{subsec:vergleichMitJava} ein Vergleich mit Java durchgeführt.
In Abschnitt~\ref{subsec:typisierung} werden Unterschiede in der Typisierung betrachtet.
In Unterabschnitt~\ref{subsec:besonderheiten} werden neue Sprachelemente und Besonderheiten dargestellt.

Bei Groovy handelt es sich primär um eine Skriptsprache.
Groovy-Programme sollten somit einfach auszuführen sein, möglichst ohne Zwischenschritte wie etwa ein Aufruf eines Compilers.
Dies ist mit dem \code{groovy}-Programm möglich, welches eine gegebene Datei direkt als Groovy-Skript ausführt.
Ähnlich wie bei Bash oder Python können Groovy-Dateien (\code{.groovy}) aber auch mit einer Shebang\footnote{Auf UNIX-basierten Betriebssystem geben die Zeichen ``\#!'' am Anfang einer Datei an, dass diese mit dem darauffolgenden Befehl ausgeführt werden soll.}-Zeile beginnen, um sie direkt ausführbar zu machen.
Variablen und Anweisungen können direkt in der obersten Ebene platziert werden, ebenso Imports, Klassen und Methoden.
Listing~\ref{lst:hello-world.groovy} zeigt ein einfaches Hello-World-Programm, welches sich dies zunutze macht.

\codelisting{groovy}{hello-world.groovy}{Hello World in Groovy}

Speichert man das Beispiel in einer Datei namens \plain{hello-world.groovy}, kann es mit dem Befehl \code{groovy hello-world.groovy} ausgeführt werden.
Ist die Datei als ausführbar markiert, genügt durch das Shebang der Befehl \code{./hello-world.groovy}.
Die Ausgabe ist in beiden Fällen \outquote{Hello World!}.

Zu beachten ist, dass es sich zwar um ein Skript handelt, aber die Groovy-Laufzeit diese vor der Ausführung immer vollständig kompiliert.
Bei Skripts geschieht dieser Schritt jedoch ohne dass zusätzliche Dateien gespeichert werden;
der generierte Bytecode wird nur zwischengespeichert und nach der Ausführung verworfen.
Dies verstärkt den Kritikpunkt von JVM-Sprachen, lange Startzeiten zu haben, mit einer zusätzlichen Verzögerung.

Eine Groovy-Installation liefert neben Interpreter, Compiler und anderen Tools auch das \plain{groovysh}-Programm aus.
Es ermöglicht die interaktive Eingabe und sofortige Auswertung von Deklarationen, Anweisungen und Ausdrücken, ähnlich wie mit Python oder Bash.
Das Ausprobieren von kurzen Codezeilen wird dadurch vereinfacht, was besonders beim Erlernen der Sprache nützlich ist.

\subsection{Vergleich mit Java}\label{subsec:vergleichMitJava}

Da ein Großteil der Sprachelemente und deren Syntax direkt übernommen wurde, können viele Java-Programme direkt als Groovy-Programm verwendet werden.
Dies hat einerseits den Vorteil, dass bei der Überführung eines Projekts von Java zu Groovy zunächst nur wenige Änderungen vorgenommen werden müssen.
Andererseits macht es das Erlernen der Sprache für Entwickler mit einem Java-Hintergrund einfacher, da sie direkt gültigen Groovy-Code schreiben können und sich mit der Zeit spezifische Konzepte für kompakten und idiomatischen Code aneignen können.

In einigen Fällen ist ein gültiges Java-Programm jedoch nicht gleich ein gültiges Groovy-Programm.
Triviale Beispiele sind jene Programme, in denen Schlüsselwörter von Groovy als Bezeichner verwendet werden.
Zu den Kontrollstrukturen, die Groovy in Version 2 nicht unterstützt, gehören Do-While-Schleifen, Try-With-Resource- Blöcke und Block-Statements.
Die Java-Syntax für Lambda-Ausdrücke, Methoden-Referenzen und Array-Initialisierung werden von dieser Version auch nicht unterstützt.
In Groovy 3~\cite{groovy-lang:release3}, welches sich derzeit in Entwicklung befindet, werden jedoch mithilfe eines neuen Parsers Bemühungen betrieben, auch diese Sprachelemente zu erlauben und somit den Umstieg weiter zu erleichtern.

Unterschiede können auch bei Programmen auftreten, die in beiden Sprachen syntaktisch gültig sind, da sich in einigen Fällen die Semantik in Groovy anders als in Java verhält.
Dazu gehören das implizite Importieren von Pakete zusätzlich zu \code{java.lang}, darunter u.a.~\code{java.io}, \code{java.net}, \code{java.util}, \code{groovy.lang} und \code{groovy.util}, sowie die Klassen \code{BigInteger} und \code{BigDecimal} aus \code{java.math}.
Skripts, die auf Dateien oder das Internet zugreifen oder genaue Berechnungen durchführen, kommen also ohne zusätzliche Import-Deklarationen aus und sind dementsprechend kürzer.
Weiterhin verhalten sich innere und geschachtelte Klassen, String- und Char-Literale und Typkonvertierungen zwischen primitiven Typen, Wrapper-Typen, \code{BigInteger} und \code{BigDecimal} anders als in Java.
Der Zugriff auf Attribute erfolgt in Groovy standardmäßig durch Aufruf des Getters oder Setters, welche auch automatisch generiert werden können.
Weitere Informationen und Beispiele zu den oben genannten Unterschieden sind bei~\cite[Abs~3.2.]{groovy-lang:documentation} sowie in den nachfolgenden Unterabschnitten zu finden.

\subsection{Typisierung}\label{subsec:typisierung}

Das Typsystem von Groovy ist standardmäßig dynamisch und unterstützt neben nominaler auch Duck-Typisierung\footnote{Paradigma, bei dem nicht der Typ eines Objekts entscheidend ist, sondern was dieses kann --- beispielsweise welche Methoden und Attribute es bereitstellt.}~\cite[Abs.~1.6.6.]{groovy-lang:documentation}.
An vielen Stellen werden implizite Typkonvertierung, beispielsweise von und zu Strings oder Booleans durchgeführt~\cite[Abs.~1.6.3.]{groovy-lang:documentation}.
Typinformationen im Code dienen dem Compiler nur als Hilfe um Warnungen auszugeben; Fehler treten jedoch erst zur Laufzeit auf.
Wenn gewünscht ist es aber auch möglich, Typfehler zur Compilezeit zu erzwingen; dafür können die \code{@TypeChecked}- und \code{@CompileStatic}-Annotationen verwendet werden~\cite[Abs.~1.6.6.]{groovy-lang:documentation}.

Ein Unterschied zu Java, der mit der dynamischen Typisierung in Verbindung steht, ist die Auflösung von Methodenaufrufen.
In Java werden diese dynamisch anhand des Empfängertyps (Überschreiben), aber statisch anhand der Argumenttypen (Überladen) aufgelöst.
Bei Groovy sind auch die Laufzeittypen der Argument für die Auswahl entscheidend, was unter dem Begriff Multi-Methoden bekannt ist.
Listing~\ref{lst:dynamic-typing.groovy} illustriert das dynamische Verhalten.

\codelisting{groovy}{dynamic-typing.groovy}{Dynamische und Duck-Typisierung und Multi-Methoden}

Zunächst werden zwei Klassen \code{Duck} und \code{Frog} jeweils mit einer \code{quack}-Methode deklariert.
In Zeilen 3 und 4 werden außerdem zwei überladene Methoden \code{speak} definiert.
Die ForEach-Schleife in Zeile 5 iteriert nun über drei Objekte, eine \code{Duck}, einen \code{Frog} und einen String.
Die Variable \code{o} hat durch das \code{def}-Schlüsselwort den statischen Typ \code{Object}.

Im Rumpf der Schleife wird die überladene \code{speak}-Funktion aufgerufen.
Bei den ersten zwei Iterationen enthält \code{o} eine \code{Duck}- bzw. eine \code{Frog}-Instanz, weshalb sich der Aufruf auf die allgemeine \code{speak(Object)}-Implementierung bezieht.
Da das Argument zur Laufzeit eine Methode \code{quack} bereitstellt, kann diese (dynamisch) aufgerufen werden, obwohl \code{Object} die Methode selber nicht hat.
Die Ausgabe ist also \outquote{I'm a duck} gefolgt von \outquote{I'm a frog}.

In der dritten Iteration enthält die Variable \code{o} einen String.
Trotz des statischen Typs \code{Object} wählt die Groovy-Laufzeitumgebung die \code{speak(String)}-Funktion aus, da die Auflösung überladener Methoden auch anhand der dynamischen Argumenttypen erfolgt.
Die letzte Zeile der Ausgabe ist also \outquote{A String can't talk}.
In Java würde die Methodenauswahl zur Kompilierungszeit erfolgen und immer die \code{speak(Object)}-Variante aufrufen, da dort beim Überladen nur der statische Typ relevant ist.

\subsection{Besonderheiten}\label{subsec:besonderheiten}

Trotz der Anlehnung an Java bringt Groovy auch einige Besonderheiten mit sich.
Selbst elementare Sprachkonzepte wie Strings, Zahlen und boolsche Werte sind vielfältiger.
Listen, Maps und andere Datenstrukturen sind einfacher zu verwenden; auch Null-Sicherheit wird durch die Sprache erleichtert.
Optionale Syntax und Closures ermöglichen es, DSLs zu definieren und verwenden.
Auf die oben genannten Sprachkonzepte wird in diesem Unterabschnitt genauer eingegangen.
Zuletzt werden einige Elemente aus der Standardbibliothek betrachtet, die weiterführende Anwendungen wie Dateiverarbeitung ermöglichen.

\subsubsection{Zahlen und Operatoren}\label{subsubsec:zahlen}

Mathematische Konzepte haben eine besondere Bedeutung in Groovy.
Der einfache Gebrauch von \break\code{BigInteger} und \code{BigDecimal} ist für wissenschaftliche und finanzielle Berechnungen interessant.
Auch die Möglichkeit, Operatoren zu überladen, ist dabei relevant.
In Listing~\ref{lst:numbers.groovy} werden diese Besonderheiten beispielhaft dargestellt.

\codelisting{groovy}{numbers.groovy}{Rechnen mit Zahlen und eigenen Klassen}

In Zeile 1 wird das klassische Beispiel für Fließkommafehler gezeigt, welches in Groovy keine Probleme verursacht, da Literale mit Nachkommastellen ohne explizites Suffix oder Cast vom Typ \code{BigDecimal} sind.
Das Ergebnis von Divisionen von zwei ganzzahligen Werten wird standardmäßig nicht gerundet.
Der Operator \code{**} kann verwendet werden, um die Potenz einer Zahl zu berechnen.

Übliche Operatoren können auch von eigenen Klassen verwendet werden, indem sie Methoden mit besonderen Namen wie \code{plus} für \code{+}, \code{leftShift} für \code{<<} oder \code{next} für \code{++} deklarieren.
Dies macht es möglich, Klassen für mathematische Objekte wie komplexe Zahlen, Brüche oder Polynome zu erstellen und mit Operatoren auszustatten, als wären sie Teil der Sprache.
Eigene Operatorsymbole können jedoch nicht definiert werden.
In Zeile 2 bis 5 ist eine einfache Klasse \code{MyNum} deklariert\footnote{Die Annotation \code{@groovy.transform.Canonical} sorgt dafür, dass ein Konstruktor sowie Standardimplementierungen der Methoden \code{equals}, \code{hashCode} und \code{toString} generiert werden.}, welche durch die \code{plus}-Methode die Addition ihrer Instanzen erlaubt.

In Groovy verhalten sich primitive Datentypen zu Kompilierungs- und Laufzeit wie Objekte ihrer Wrapper-Klassen~\cite[Abs.~3.2.10.]{groovy-lang:documentation}.
Somit kann man mit ihnen alles machen, was auch bei Objekten möglich ist, beispielsweise Attributzugriff oder Methodenaufrufe.
Das Verhalten von Zahlen und Operatoren ist unter~\cite[Abs.~1.1.5.]{groovy-lang:documentation} und~\cite[Abs.~1.2.]{groovy-lang:documentation} genauer dokumentiert.

\subsubsection{Zeichenketten}\label{subsubsec:strings}

Auch Zeichenketten sind vielfältiger als in Java.
Es existieren sechs Arten von String-Literalen in Groovy:

\begin{itemize}\setlength{\itemsep}{0pt}\setlength{\parskip}{0pt}\setlength{\parsep}{0pt}
\item \code{'single quoted'}, unterstützt keine Interpolation oder Zeilenumbrüche,
\item \code{'''triple single quoted'''}, unterstützt keine Interpolation aber mehrere Zeilen,
\item \code{"double quoted"}, unterstützt Interpolation mit dem \code{$}-Zeichen, aber nur einzeilig,
\item \code{"""triple quoted"""}, unterstützt Interpolation und mehrere Zeilen,
\item \code{/slashy/}, für reguläre Ausdrücke, unterstützt auch Interpolation und Zeilenumbrüche, und
\item \code{$/dollar slashy/$}, ähnlich zu \code{/slashy/} aber mit besonderen Escape-Regeln.
\end{itemize}

Besonders interessant ist hierbei die Interpolation, welche ein einfaches Einsetzen von Werten in Strings erlaubt.
Listing~\ref{lst:string-interpolation.groovy} zeigt, wie diese in String-Literalen zum Einsatz kommt.

\codelisting{groovy}{string-interpolation.groovy}{String-Interpolation und Matching mit regulären Ausdrücken}

Im ersten String-Literal in Zeile 2 wird anstelle von \code{$a} und \code{$b} der Wert der Variablen eingesetzt;
\code{${a+b}} wertet den Ausdruck \code{a+b} aus und setzt das Ergebnis in den resultierenden String ein.

In Zeile 3 wird ein Slashy-String-Literal zusammen mit dem Find-Operator \code{=~} verwendet, um einen String mit einem regulären Ausdruck zu matchen.
Im Slashy-String müssen die Escape-Zeichen~\plain{\ } nicht doppelt geschrieben werden; der reguläre Ausdruck wird somit leichter lesbar.
Das Ergebnis des \code{=~}-Operators ist ein \code{Matcher}.
Dieser kann implizit zu einem Bool konvertiert werden, der den Erfolg angibt, und ermöglicht den Zugriff auf die Gruppen des regulären Ausdrucks wie in Zeile 5.

\subsubsection{Listen und Maps}\label{subsubsec:listen-und-maps}

Auch Listen und Maps sind fest in der Syntax von Groovy verankert und haben einen großen Satz von Funktionen, um ihre Verwendung zu erleichtern.
Listing~\ref{lst:lists-and-maps.groovy} illustriert einige Möglichkeiten, Listen und Maps zu verwenden.

\codelisting{groovy}{lists-and-maps.groovy}{Erstellen, Zugriff und Verändern von Listen und Maps}

In Zeile 1 wird zunächst eine Liste mit der \code{[]}-Syntax angelegt;
diese ist ohne explizite Typangabe eine \code{ArrayList}.
In Zeile 2 wird der Index-Operator verwendet, um auf das erste und letzte Element zuzugreifen, dabei ist die Verwendung negativer Zahlen wie in Python möglich.
Danach wird mit dem \code{<<}-Operator ein Element ans Ende der Liste angehängt.
Die 4.~Zeile zeigt, dass der Index-Operator auch mehrere Argumente sowie Intervalle unterstützt.
Letztere sind mit dem geschlossenen \code{..}-Operator sowie dem halboffenen \code{..<}-Operator auch in anderen Kontexten definiertbar, beispielsweise in ForEach-Schleifen.

In Zeile 6 wird eine Map angelegt, zu erkennen an den Schlüssel-Wert-Paaren in den eckigen Klammern.
Bei den Bezeichnern \code{a} und \code{b} handelt es sich nicht um Variablennamen, stattdessen wird hier die Zeichenkette \code{a} bzw.~\code{b} als Schlüssel verwendet.
Das erzeugte Objekt verwendet die \code{LinkedHashMap}-Implementierung, damit die Reihenfolge der Schlüssel erhalten bleibt.
Zeile 7 zeigt, wie der Zugriff auf den Wert eines Schlüssels sowohl mit dem Index-Operator als auch mit dem Punkt-Operator möglich ist, als wäre \code{d} eine Klassenvariable.
In der nächsten Zeile werden auch beide Möglichkeiten verwendet, um neue Einträge hinzuzufügen.
Auffällig ist der Zugriff auf unbekannte Schlüssel in Zeile 9, welcher auch mit dem Punkt-Operator möglich ist und in beiden Fällen \code{null} zurückgibt.

Andere Collections und Arrays können durch Coercion von Listen- oder Map-Literalen erzeugt werden.
Weitere Informationen zu Arrays, Listen und Maps sind in~\cite[Abs.~1.1.7.-1.1.9.]{groovy-lang:documentation} zu finden.

\subsubsection{Null-Sicherheit}

Obwohl Groovy im Kern eine Skriptsprache ist und Sicherheit daher eine kleinere Rolle spielt, ist es dennoch im Bereich Null-Sicherheit besser ausgestattet als Java.
Im Kern stehen dabei zwei sehr nützliche Operatoren, \code{?.} und \code{?:}, welche im Folgenden kurz vorgestellt werden.

Der Safe-Navigation-Operator \code{?.} wird verwendet, um \code{NullPointerException}s zu vermeiden.
Anstelle des \code{.}-Operators für Methodenaufrufe und Attributzugriffe sorgt er dafür, dass das Ergebnis des Aufrufs oder Zugriffs \code{null} ist, wenn der Empfänger \code{null} ist.
Somit ist beispielsweise der Wert von \code{person?.name} gleich \code{null}, wenn \code{person} schon \code{null} ist.

Der \code{?:}-Operator, auch Elvis-Operator genannt, ist syntaktischer Zucker, der Ausdrücke der Form \code{lhs ?: rhs} durch \code{lhs ? lhs : rhs} ersetzt.
Dies erscheint im ersten Moment nicht nützlich, doch durch die Regeln der Groovy Truth~\cite[Abs.~1.6.5.]{groovy-lang:documentation}, welche die implizite Konvertierung zu Bool-Werten bestimmen, ist er sehr mächtig.
Eine dieser Regeln besagt, dass \code{null}-Referenzen als \code{falsy}\footnote{In Skriptsprachen wird dieser Begriff häufig für Werte verwendet, die sich in boolschen Ausdrücken wie \code{false} verhalten, aber eigentlich kein boolscher Wert sind.} gelten.
Somit wird \code{rhs} vom Ternary-Operator ausgewählt, wenn \code{lhs} \code{null} ist, und \code{lhs} wenn nicht.
Jedoch können die Regeln auch zu Problemen in der Semantik führen:
Leere Strings, Listen und Maps oder Zahlen mit dem Wert \code{0} gelten neben \code{null}-Referenzen auch als \code{falsy}.
Der Elvis-Operators gibt mit solchen Werten als linkem Operand auch den rechten Operand zurück.

\subsubsection{Closures}\label{subsubsec:closures}

Konzeptuell sind Closures ähnlich zu Lambda-Ausdrücken in Java, indem sie eine Möglichkeit bieten, Anweisungen wie ein Objekt zu speichern und zu übergeben.
Sie sind sowohl für das Schreiben von kompaktem Code als auch für DSLs unerlässlich.
Listing~\ref{lst:closures-fp.groovy} enthält zunächst einige Beispiele, wie sie für funktionale Operationen auf Listen verwendet werden, welche mit Schleifen o.ä.~erheblich umfangreicher wären.

\codelisting{groovy}{closures-fp.groovy}{Closures für funktionale Listenoperationen}

Zunächst werden in Zeile 1 zwei Closures an Variablen gebunden.
\code{doubler} verdoppelt den Wert des impliziten Parameters \code{it}, während \code{adder} zwei Parameter addiert.
Die Variablen können wie Funktionen aufgerufen werden, wie in Zeile 2 zu sehen ist.
Sie können jedoch auch selbst als Argument an andere Methoden übergeben werden.
So wird in Zeile 3 mit \code{collect} eine neue Liste durch elementweises Anwenden von \code{doubler} erstellt.
Mit \code{inject}, welches einer Reduktion entspricht, kann mit Startwert 0 und \code{adder} die Summe einer Liste berechnet werden.
Zeile 4 zeigt, dass Closures auch direkt übergeben werden, hier zur Ermittlung der geraden Zahlen zwischen 1 und 7.
Die Operationen \code{collect}, \code{findAll} und \code{inject} sind auf allen Collection-ähnlichen Klassen vorhanden.
In funktionalen Programmiersprachen sowie in der Java-\code{Stream}-API sind sie jedoch meist als \code{map}, \code{filter} und \code{reduce} bzw.~\code{fold} bekannt.

Eine Besonderheit von Closures ist, dass Methodenaufrufe und Feldzugriffe im Rumpf nicht an der Stelle verfügbar sein müssen, wo die Closure erstellt wird.
Dafür wird der \code{delegate}-Mechanismus bereitgestellt, der sehr gut für DSLs geeignet ist und in Listing~\ref{lst:closures-dsl.groovy} gezeigt wird.

\codelisting{groovy}{closures-dsl.groovy}{Verwendung von Closure Delegates für eine Rechenmaschinen-DSL}

Das Programm beginnt mit der Deklaration der Funktion \code{eval}, welche eine Closure akzeptiert und einen Variable \code{result} speichert.
Danach werden auf dem \code{delegate} der Closure die Methoden \code{add} und \code{mul} durch weitere Closures definiert.
Diese verändern entsprechend die \code{result}-Variable mit dem übergebenen Argument \code{it}.
In Zeile 4 wird die modifizierte Closure aufgerufen und der Wert der Ergebnisvariable ausgegeben.
Die Verwendung der dadurch definierten DSL, einer einfachen Rechenmaschine, folgt in Zeile 6.
Die Closure, die der \code{eval}-Funktion übergeben wird, bezieht sich auf die in diesem Kontext unbekannten Funktionen \code{add} und \code{mul}.
Diese müssen jedoch erst dann (dynamisch) aufgelöst werden, wenn die Closure aufgerufen wird.
Da zu diesem Zeitpunkt schon das \code{delegate} verändert ist, können die bisher unbekannten Funktionen nun über dieses aufgelöst werden, als hätte man \code{delegate.add/mul(...)} aufgerufen.
Somit kann das Programm erfolgreich ausgeführt werden und gibt \outquote{5.5} aus.

\subsubsection{Optionale Syntax}\label{subsubsec:optionale-syntax}

Groovy erlaubt es, Punkt-Operatoren und Klammern bei Methodenaufrufen wegzulassen.
Dies ermöglicht die einfache Erstellung von nahezu natürlichsprachigen DSLs, wie es Listing~\ref{lst:optional-parens.groovy} zeigt.

\codelisting{groovy}{optional-parens.groovy}{Eine weitere DSL mithilfe von optionalen Klammern und Punkten}

Hier wird zunächst eine Hilfsfunktion \code{make} erstellt, welche die DSL definiert.
Um das Beispiel kurz zu halten, werden hierfür Map-Literale und Closures verwendet, die in den Unterabschnitten~\ref{subsubsec:listen-und-maps} und~\ref{subsubsec:closures} erklärt wurden.
Bei echten DSLs sollten stattdessen Klassen und Methoden verwendet werden.
Hier wurde zusätzlich ausgenutzt, dass die Verwendung des \code{return}-Schlüsselworts auch optional ist.
In Zeile 4 werden aus Illustrationsgründung symbolische Variablen für einige Zeichenketten definiert.
Danach folgt die Verwendung der definierten DSL:
Der Befehl sieht aus wie ein englischer Satz, ist aber semantisch äquivalent zu der Anweisung in Zeile 6.
Bei den aufeinanderfolgenden Bezeichnern werden abwechselnd Punkt-Zugriffe und Methodenaufrufe impliziert.
Die Ausgabe ist zweimal der Satz \outquote{My cake contains sugar, eggs and flour.}.

Es fällt auch auf, dass bei diesem und anderen Listings an Zeilenenden keine Semikolons gebraucht werden, da auch diese in den meisten Fällen optional sind.
In Listing~\ref{lst:closures-dsl.groovy}, Zeile 7, ist zu beobachten, wie mit einem expliziten Semikolon auch mehrere Anweisungen in einer Zeile stehen können.

\subsubsection{Standardbibliothek und Erweiterungen}\label{subsubsec:standard-bibliothek}

Groovy baut auf dem Java Development Kit (JDK) auf, stellt aber auch seine eigene Klassen bereit.
Kollektiv wird die Standardbibliothek als GDK bezeichnet.
Besonders ist, dass dieses auch bestehende Klassen aus dem JDK erweitert.

Klassen wie \code{List} und andere Collections haben in Groovy die in Unterabschnitt~\ref{subsubsec:listen-und-maps} verwendeten funktionalen Operationen.
Die Klasse \code{String} wird dabei auch als Collection (von Zeichen) behandelt und unterstützt ähnliche Operationen.
\code{File} und andere IO-Klassen besitzen Erweiterungen für das vereinfachte Lesen und Schreiben.
Auch Verzeichnisbäume können innerhalb weniger Zeilen mit Closures traversiert werden.
Durch Möglichkeiten zur Ausführung von Kommandozeilen kann Groovy auch als Klebersprache verwendet werden.
Da diese Erweiterungsmethoden meist Exceptions verstecken, kann auf Try-Catch-Blöcke verzichtet werden.
Somit ist auch die Dateiverarbeitung in kurzen Skripts möglich.

Zudem ist es möglich, selber bestehende Klassen zu verändern.
Durch den \code{metaClass}-Mechanismus können beispielsweise Methoden zu Klassen oder einzelnen Instanzen hinzugefügt werden.
Auch Properties können von speziellen Erweiterungen dynamisch aufgelöst werden, was insbesondere beim Zugriff auf Maps wie in Unterabschnitt~\ref{subsubsec:listen-und-maps} Anwendung findet.
Weitere Konzepte der Metaprogrammierung, die teilweise von Ruby inspiriert sind, sind in~\cite[Abs.~3.4.]{groovy-lang:documentation} zu finden.

In vielen Teilen der Standardbibliothek spielen Closures eine Schlüsselrolle.
So existieren Klassen, die geschachtelte Closures und dynamische Methodennamen zum Generieren von JSON, HTML, XML und sogar Swing-GUIs verwenden.
Diese sind zusammenfassend unter der Bezeichnung ``Builder'' bekannt, werden hier aber nicht näher betrachtet~\cite[Abs.~3.15.8]{groovy-lang:documentation}.

% --------------- ---------------

\section{Community, Einsatz und Verbreitung}\label{sec:community}

Nach dem technischen Teil zur Sprache soll nun auch das Ökosystem um Groovy betrachtet werden.
Im Januar 2019 ist Groovy auf Platz 21 des TIOBE Index~\cite{tiobe-index} und somit die zweitbeliebteste JVM-Sprache, nach Java auf dem ersten Platz.
Die Konkurrenz bilden dabei Scala und Kotlin auf Platz 28 und 31, es handelt sich bei diesen jedoch nicht um Skriptsprachen.
Auf StackOverflow~\cite{stackoverflow:groovy} sind über 22.000 Fragen und Antworten zu Groovy finden.
Durch die Quelloffenheit der Groovy-Implementierung auf GitHub~\cite{github:groovy} kann sich jeder an der Weiterentwicklung und Fehlerbehebung beteiligen.
Da theoretisch jede Java-Bibliothek auch von Groovy-Code verwendbar ist, können sich Entwickler auf über 3,4 Mio.~Jar-Artefakte auf Maven Central~\cite{maven-central} verlassen.

Doch auch in Groovy selbst sind große Projekte entstanden.
Im Folgenden werden kurz zwei dieser Projekte vorgestellt --- Gradle und Grails.
Ein weiteres Framework, Griffon, dient zum Erstellen von Desktop-Anwendungen basierend auf Swing und JavaFX, soll aber nicht näher erläutert werden.
Zudem existiert das Test-Framework Spock, welches DSLs für Testfälle und Verhaltensspezifikation nutzt und auf JUnit aufbaut.

Auch für finanzielle und Business-Anwendungen ist Groovy ausgelegt, exakte Berechnung und natürlichsprachige DSLs für Businessregeln kommen dabei zum Einsatz.
Ein nennenswertes Einsatzbeispiel aus dem Verwaltungsbereich kommt vom Europäischen Patentamt.
Dieses hat mit Groovy eine Platform zum automatischen Verteilen von Patentinformationen an andere Standorte entwickelt~\cite{epo}.

\subsection{Gradle}\label{subsec:gradle}

Gradle~\cite{gradle} ist ein Open-Source-Tool zur Build-Automatisierung, welches in Groovy und Java geschrieben wurde.
Konfiguriert wird es durch Groovy-Skripte, die sich die DSL-Fähigkeiten der Sprache zunutze machen.
Dies hebt es von seinen Vorgängern Apache Ant und Apache Maven, welche XML-Dateien verwenden, durch erhöhte Nutzerfreundlichkeit ab.

Im Konfigurationsskript können Module und deren Code- und Jar-Abhängigkeiten definiert werden.
Deren Herunterladen, Cachen und Verlinken wird automatisch gehandhabt.
Aufgaben (Tasks), die ihrerseits auch Abhängigkeiten haben können, können selber programmiert werden oder vordefinierte Abläufe kombinieren.
Zu letzteren zählen beispielsweise das Kopieren, Kompilieren, Verpacken, Signieren und Veröffentlichen.
Gradle ist besonders für große Projekte gut geeignet, da durch inkrementelle Builds nur geänderte Dateien verarbeitet werden müssen.

Seit dem ersten Release 2005 wurde Gradle stets weiterentwickelt.
Durch Plugins werden neben Java und Groovy auch andere Zielsprachen wie Scala und sogar Python und C++ unterstützt.
Unter anderem wird Gradle bei Adobe, LinkedIn und Netflix sowie für Android-Anwendungen eingesetzt.

\subsection{Grails}\label{subsec:grails}

Grails~\cite{grails} ist ein in Groovy geschriebenes Web-Framework, welches ebenfalls quelloffen ist.
Der ursprüngliche Name ``Groovy on Grails'' entstand in Anlehnung an das populäre Ruby-on-Rails-Framework, von dem einige Konzepte übernommen wurden.
Es basiert auf dem Backend-Framework Spring Boot, setzt aber wie Rails auf Konvention statt Konfiguration, um einen leichteren Einstieg zu erlauben.
Schnittstellen für Datenbankzugriff und JSON/REST-APIs sowie ein Templatesystem für HTML-Seiten werden bereitgestellt.
Dabei finden auch die in Unterabschnitt~\ref{subsubsec:standard-bibliothek} genannten JSON- und HTML-Builder der Standardbibliothek Anwendung.
Das Framework nutzt die DSL-Unterstützung von Groovy unter anderem für Validierung, Datenbankanfragen und Textformattierung.
Grails wirbt mit dem Einsatz bei vielen großen Firmen, darunter Google, IBM, LinkedIn, Mastercard und Oracle.

% --------------- ---------------

\section{Zusammenfassung}\label{sec:zusammenfassung}

Es hat sich gezeigt, dass Groovy da ansetzt, wo Java aufhört --- kurze und prägnante Programme, die mit wenigen Zeilen viel bewirken können.
Anweisungen auf oberster Ebene, optionale und dynamische Typen und das Auslassen des Kompilierungsschritts ermöglichen schnell einsatzbereite Skripte, wie sie von Ruby, Python oder Bash bekannt sind.
Die Groovy Shell rundet dies mit einer interaktiven Umgebung ab und vereinfacht das Erlernen.

Die Sprache gibt dem Entwickler viele Freiheiten bei der Syntax durch Closures und optionalen Zeichen, wodurch mächtige DSLs erstellt und optisch anschaulich genutzt werden können.
Trotz der Skriptnatur ist, wenn gewünscht, auch für Sicherheit gesorgt:
Die in Java häufig kritisierte Nullsicherheit ist in Groovy besser gelöst;
Listen und Maps sind sicherer und einfacher zu verwenden als Arrays.
Typ- und Performanceprobleme, besonders bei großen Projekten, können durch die optionale statische Kompilierung umgangen werden,
die bei JVM-Sprachen inherenten Probleme der Startup-Zeit und des Speicherverbrauchs jedoch nicht.

Dennoch profitiert Groovy von dieser Zielplatform:
Der leichte Umstieg von Java sowie der Einsatz mit Gradle fördern die Bekanntheit der Skriptsprache.
Die Möglichkeit, die sehr umfangreiche Anzahl von Java-Bibliotheken zu nutzen, sind weitere wichtige Faktoren.
Durch die Standardisierung, die Zugehörigkeit zur Apache Foundation und die Quelloffenheit ist Groovy eine robuste und dennoch stets wachsende Sprache.
Mit Groovy 3 ist die Sprache noch immer lebendig und versucht, den Umstieg weiter zu vereinfachen.
Projekte wie Gradle und Grails unterstützen den Einsatz von Groovy in den populärer werdenden DevOps- und Web-Anwendungsbereichen.

Insgesamt ist Groovy eine Sprache, die den Entwickler weder einschränkt noch belastet.
Es besteht freie Paradigmenwahl:
Groovy ist objektorientiert und funktional, dynamisch und statisch, imperativ und deklarativ, interpretiert und kompiliert.
Es bietet ein Rahmenwerk für sehr kurze aber auch sehr lange Programme.
Wo Java für kurze und Skriptsprachen für lange Programme nicht geeignet sind, setzt Groovy ein und verbindet zwei Welten.

% --------------- ---------------

\bibliographystyle{ieeetr}
\begingroup
	\linespread{0}\selectfont % condense the bibliography entries
	\bibliography{main}
\endgroup

% --------------- ---------------
\end{document}
