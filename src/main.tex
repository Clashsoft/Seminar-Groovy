%%
%% Author: adriankunz
%% 2018-12-28
%%

% =============== Preamble ===============

% --------------- Definitions ---------------
\newcommand{\paperTitle}[0]{Die Sprache Groovy}
\newcommand{\paperSubtitle}[0]{Seminar Skriptsprachen, Wintersemester 2018/19}
\newcommand{\paperDate}[0]{\today}
\newcommand{\paperKeywords}[0]{Groovy, Seminar, Skriptsprachen}
\newcommand{\paperAuthor}[0]{Adrian Kunz}

% --------------- Commands ---------------
\newcommand{\code}[1]{\mintinline{java}{#1}}

% --------------- Document Setup ---------------
\documentclass[11pt,a4paper]{article}
\usepackage[left=2.5cm,right=2.5cm,top=2.5cm,bottom=2cm]{geometry}

\title{\paperTitle}
\author{\paperAuthor}
\date{\paperDate}

% --------------- Packages ---------------
\usepackage[ngerman]{babel}
\usepackage[T1]{fontenc}
\usepackage[utf8]{inputenc}
\usepackage{url}

% Minted
\usepackage{minted}
\usemintedstyle{vs}
\setminted{frame=single,framesep=2mm,tabsize=2}

% Hyperref
\usepackage{hyperref}
\hypersetup{
pdftitle={\paperTitle{}},
pdfauthor={\paperAuthor{}},
pdfsubject={\paperTitle{}},
pdfkeywords={\paperKeywords},
bookmarksnumbered=true,
bookmarksopen=true,
hidelinks,
}
\usepackage{hypcap}

% =============== Document ===============

\begin{document}



	\subsection{Besonderheiten}\label{subsec:besonderheiten}

	% --------------- ---------------

	\section{Community, Verwendung und Verbreitung}\label{sec:community}

	Groovy wird von vielen Unternehmen verwendet, unter anderen Google, IBM und LinkedIn~\cite{GroovyLangOrg,wiki:Groovy}.

	\subsection{Gradle}\label{subsec:gradle}

	\subsection{Grails}\label{subsec:grails}

	% --------------- ---------------

	\section{Zusammenfassung}\label{sec:zusammenfassung}

	% --------------- ---------------

	\bibliography{main}
	\bibliographystyle{ieeetr}

	% --------------- ---------------
\end{document}
